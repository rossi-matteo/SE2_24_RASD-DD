\section{Scope}
\label{sec:scope}%
\textbf{Students\&Companies} (S\&C) is a platform that acts as an intermediary system facilitating the internship matching process between students and companies. It allows companies to advertise internships and students to search, receive customized recommendations, and initiate contact.

The platform defines \textbf{Recommendation} as the automated process of identifying suitable internship opportunities for students and potential candidates for companies. Following this, a \textbf{Contact} represents the phase in which students and companies communicate via the platform to conduct the selection process, including interviews and candidate selection.

The system automates key activities such as generating recommendations using various mechanisms, coordinating interviews, and collecting feedback to improve its algorithms. Additionally, it provides tools to monitor the progress of contacts and internships, manage issues through communication features, and enable universities to supervise the status of internships, ensuring compliance and resolving possible complaints effectively.

\section{Definitions, Acronyms, Abbreviations}
\label{sec:definition_acronyms_abbreviations}

\begin{itemize}

\item \textbf{System, Platform}: these terms are used interchangeably when referring to the system-to-be-developed.

\item \textbf{University, Company}: when the terms are referenced as performing an action, it means that the action is executed by a representative acting on behalf of the respective entity.

\item \textbf{Party}: the term refers to the entities actively involved in the process of applying, participating and managing internships, so both Student and Company; it doesn't include the University.

\end{itemize}

\begin{table}[H]
    \begin{center}
        \begin{tabular}{ |l|l| }
            \hline
            \textbf{Acronyms} & \textbf{Definition}                              \\
            \hline
            RASD           & Requirements Analysis \& Specification Document   \\
            \hline
            DD             & Design Document   \\
            \hline
            API            & Application Programming Interface                  \\
            \hline
            REST           & REpresentational State Transfer   \\
            \hline
            DDD            & Domain-Driven Development   \\
            \hline
            DBMS           & DataBase Management System   \\
            \hline
            LLM            & Large Language Model  \\
            \hline 
            JSON           & JavaScript Object Notation  \\
            \hline
            URL            & Uniform Resource Location  \\
            \hline
            URI            & Uniform Resource Identifier  \\
            \hline
            HTTPS          & HyperText Transfer Protocol over Secure Socket Layer   \\
            \hline
            SMTP           & Simple Mail Transfer Protocol  \\
            \hline
            UI             & User Interfaces \\
            \hline
            CDC            & Change Data Capture \\
            \hline
            QoS             & Quality of Service \\
            \hline            
            E2E            & End-To-End \\
            \hline
         \end{tabular}
        \caption{Acronyms used in the document.}
        \label{tab:acronyms}%
    \end{center}
\end{table}

\begin{table}[H]
    \begin{center}
        \begin{tabular}{ |l|l| }
            \hline
            \textbf{Abbreviations} & \textbf{Definition}
            \\
            \hline
            S\&C    & Student\&Company \\
            \hline
            UC      & Use Case \\
            \hline
         \end{tabular}
        \caption{Abbreviations used in the document.}
        \label{tab:Abbreviations}%
    \end{center}
\end{table}

\section{Revision history}
\label{sec:revision_history}%
\label{sec:definition_acronyms_abbreviations}%
\begin{table}[H]
    \begin{center}
        \begin{tabular}{ |l|l|l|}
            \hline
            \textbf{Revised on} & \textbf{Version}   & \textbf{Description}                           \\
            \hline
            05/01/2025            & 1.0   &   Initial Release of the document  \\
            \hline
         \end{tabular}
         \caption{Revision history}
        \label{tab:acronyms}%
    \end{center}
\end{table}

\newcounter{bib}
\setcounter{bib}{1}
\newcommand{\cbib} {\thebib\stepcounter{bib}}

\section{Reference Documents}
\label{sec:reference_documents}
\begin{itemize}
    \item{[\cbib]} Di Nitto, Rossi, Camilli, \textit{"A.Y. 2024-2025 Software Engineering 2 Requirement Engineering and Design Project"}, 2024.
    
    \item{[\cbib]} ISO/IEC/IEEE 29148:2018, \textit{"Systems and Software Engineering — Life Cycle Processes — Requirements Engineering," International Organization for Standardization, International Electrotechnical Commission, and Institute of Electrical and Electronics Engineers}, 2018.

    \item{[\cbib]} \href{https://learn.microsoft.com/en-us/dotnet/architecture/cloud-native/distributed-data#database-per-microservice-why}{\textcolor{blue}{\textit{Microsoft Learn - Cloud-native data patterns}}}
    \label{document: cloud_native_data_patters}
    
    \item{[\cbib]} \href{https://learn.microsoft.com/en-us/previous-versions/msp-n-p/dn589795(v=pandp.10)#functional-partitioning}{\textcolor{blue}{\textit{Microsoft Learn, Data Partitioning guidance}}}
    \label{document: data_partitioning}

    \item{[\cbib]} \href{https://learn.microsoft.com/en-us/dotnet/architecture/cloud-native/relational-vs-nosql-data#the-cap-theorem}{\textcolor{blue}{\textit{Microsoft Learn, Relational vs. NoSQL data}}}
    \label{document: relational_nosql_data}

    \item{[\cbib]} \href{https://learn.microsoft.com/en-us/previous-versions/msp-n-p/dn589800(v=pandp.10)#eventual-consistency}{\textcolor{blue}{\textit{Microsoft Learn, Data Consistency primer}}}
    \label{document: data_consistency}

    \item{[\cbib]} \href{https://microservices.io/patterns/apigateway.html}{\textcolor{blue}{\textit{Chris Richardson, Microservices: API Gateway Pattern}}}
    \label{document: api_gateway}

\end{itemize}

\section{Overview}
\subsection{Document Structure}
\label{sec:doc_structure}%
This document is composed of six sections:
\begin{itemize}
    \item{} \nth{1} Chapter - Introduction
    \item{} \nth{2} Chapter - Architectural Design
    \item{} \nth{3} Chapter - User Interface Design
    \item{} \nth{4} Chapter - Requirements Traceability
    \item{} \nth{5} Chapter - Implementation, Integration and Test Plan
    \item{} \nth{6} Chapter - Effort Spent
    \item{} \nth{7} Chapter - References
\end{itemize}
